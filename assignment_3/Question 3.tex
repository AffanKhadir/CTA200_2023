
 
\documentclass[12pt]{article}
 
\usepackage[margin=1in]{geometry} 
\usepackage{amsmath,amsthm,amssymb}
\usepackage{graphicx}
\usepackage[dvipsnames]{xcolor}
\usepackage{physics}
\usepackage{caption}
\usepackage{subcaption}
\usepackage{graphicx}
\usepackage{verbatim}
\usepackage{placeins}
\newcommand{\N}{\mathbb{N}}
\newcommand{\Z}{\mathbb{Z}}
\newcommand{\R}{\mathbb{R}}
\newcommand*\ol[1]{\overline{#1}}
\graphicspath{ {C:/Users/affan/Pictures/Screenshots/} }
\DeclareMathOperator{\spn}{span}
\newcommand{\p}[1]{\left ( #1\right )}
\newcommand*\diff{\mathop{}\!\mathrm{d}}
\newcommand*{\img}[1]{%
    \raisebox{-.02\baselineskip}{%
        \includegraphics[
        height=\baselineskip,
        width=\baselineskip,
        keepaspectratio,
        ]{#1}%
    }%
}
 
\newenvironment{theorem}[2][Theorem]{\begin{trivlist}
\item[\hskip \labelsep {\bfseries #1}\hskip \labelsep {\bfseries #2.}]}{\end{trivlist}}
\newenvironment{lemma}[2][Lemma]{\begin{trivlist}
\item[\hskip \labelsep {\bfseries #1}\hskip \labelsep {\bfseries #2.}]}{\end{trivlist}}
\newenvironment{exercise}[2][Exercise]{\begin{trivlist}
\item[\hskip \labelsep {\bfseries #1}\hskip \labelsep {\bfseries #2.}]}{\end{trivlist}}
\newenvironment{problem}[2][Problem]{\begin{trivlist}
\item[\hskip \labelsep {\bfseries #1}\hskip \labelsep {\bfseries #2.}]}{\end{trivlist}}
\newenvironment{question}[2][Question]{\begin{trivlist}
\item[\hskip \labelsep {\bfseries #1}\hskip \labelsep {\bfseries #2.}]}{\end{trivlist}}
\newenvironment{corollary}[2][Corollary]{\begin{trivlist}
\item[\hskip \labelsep {\bfseries #1}\hskip \labelsep {\bfseries #2.}]}{\end{trivlist}}
 
\begin{document}
 

 
\title{Question 3}
\author{Affan Khadir\\ 
CTA 200}
\pagecolor{white}
\color{black}
 
\maketitle

\begin{question}1
We have been given the function: 
\begin{align*}
    f(c) = z^2 + c
\end{align*}
which is defines recursively, and that $c \in \{w = x+ iy: x \in [-2,2], y\in [-2, 2] \}$. We want to determine all the points for which the function stays bounded. 
\begin{lemma}1
    We claim that for all $|z_i|>2$, the function diverges
\end{lemma}
\begin{proof}
    Suppose that $|z_i|>2 $ Suppose that $|z_i| = 2 + \lambda$ where $\lambda>0 $. There are two cases that we must consider. \\
    \textbf{Case 1} When $|c|\leq 2$. We want to show that $|z_{i+k}| \geq 2 + (k+1)\lambda $ We will proceed by mathematical induction. Here, observe that: 
    \begin{align*}
        |z_{i+1}| = |z^2 + c| \geq |z^2| - |c| \geq |z|^2 - |c|\geq (2+\lambda)^2 - 2 > 2 + 2\lambda 
    \end{align*}
    In the above, we have used the fact that $|z^2|= |z|^2 $ for all complex numbers. Thus, the lemma holds true for $k=1$. 
    Assume that the lemma holds true for all $|z_{i+j}|: j\leq k$. We must now prove that it holds for $k + 1$. 
    \begin{align*}
        |z_{i+k+1}|  = |z_{i+k}^2 +c | \geq |z^2_{i+k}| -|c| \geq (2+(k+1)\lambda)^2 - 2 \geq 2 + (k+2) \lambda 
    \end{align*}
    Then, observe that: 
    \begin{align*}
        \lim_{k\to \infty} |z_{i+k}|= \lim_{k\to \infty } 2 + (k+1) \lambda =\infty
    \end{align*}
    Thus, the function diverges for these values of $c$. \\
    \textbf{Case 2}. Now, we will prove that it diverges for $|c|> 2$. Consider two possible sub-cases:\\
    (A) $2< |c| < |z_i| $: Here, we claim that $|z_{i+k}| \geq 2+ (k+1) \lambda $
    \begin{align*}
        |z_{i+1}|= |z_i^2 + c| \geq |z_i^2| - |c| \geq 2 + 2\lambda
    \end{align*}
    Following a similar argument as in \textbf{Case 1}, we can prove that our claim is true. Then: 
    \begin{align*}
        \lim_{k\to \infty} |z_{i+k} | = \lim_{k\to \infty} 2+ (k+1) \lambda = \infty
    \end{align*}
    (B) $2<|z_i|<|c|$ Here we will again split it into two further sub-cases.\\
    (i) $|z_i^2|>|c|$. Then, we can proceed by a similar argument as in the first case and get that $|z_{i+k}|\to \infty $ as $k\to \infty$. \\
    (ii) $|c| > |z_i^2| $. Assume that $|c| = |z_i^2| + \beta $ for some $\beta> 0$. Then: 
    \begin{align*}
        |z_{i+1}| = |z_i^2 + c| \geq |c| - |z_i^2| > |c| 
    \end{align*}
    Then, we will end up in case A, so the lemma holds true. \\
Thus, it has been proven that the function diverges for all $|z_i| >2$. 
\end{proof}
From this, we know that all points that we can set a rough criteria of divergence to be if some $|z_i|>2$ in the iteration. Therefore, that is the criteria of divergence that we will be using. We will be running roughly a 10 iterations, claiming that if the function doesn't diverge in ten iterations then it doesn't diverge at all. Using this criteria, we find a rough estimate for the points to be: \\

\begin{figure}[h]
    \centering
    \includegraphics{CTA200/Mandelborot.png}
    \caption{Plot displaying the set of points that remain bounded. }
    \label{fig:my_label}
\end{figure}
Now, we will find the iteration number for which the function diverges. To do this, we will find $i: |z_i|>2$. \\
\begin{figure}[H]
    \centering
    \includegraphics{CTA200/Mandelborot colour.png}
    \caption{Plot displaying the iteration number at which points diverge.}
    \label{fig:my_label}
\end{figure}
\FloatBarrier\\
\end{question}

\begin{question}2
    Here, we will be analyzing the three Fourier modes that give the Lorenz equations: 
    \begin{align*}
        \dot{X}  &= \sigma (X-Y) \\
        \dot{Y} & = rX- Y-XZ\\
        \dot{Z} & = -bZ+ XY
    \end{align*}
    Once a python function is written for $W  = (X, Y, Z)$, we can use solve\_ivp to solve the equation with the given initial conditions $W_0 = (0, 1, 0)$ and $(\sigma, r, b) = (10, 28, 8/3) $
    \begin{figure}[h]
        \centering
        \includegraphics[scale = 0.7]{CTA200/Lorenz.png}
        \caption{A plot showing the value of the $Y$ Fourier mode against the number of iterations. Here, the number of iterations is represented as $t/\delta t$, where $\delta t = 0.01$}
        \label{fig:my_label}
    \end{figure}\\
    \begin{figure}
        \centering
        \includegraphics[scale =0.7]{CTA200/Lorenz2.png}
        \caption{A plot showing the phase portrait of $Z$ against $Y$ AND $X$ against $Y$.}
        \label{fig:my_label}
    \end{figure}
    Now, we will perturb the initial conditions slightly, and observe the change that this cause in $W$
    \begin{figure}
        \centering
        \includegraphics[scale = 0.7] {CTA200/Lorenz3.png}
        \caption{A plot of the logarithmic distance between $W$ and $W' $ against time. $W$ started with initial conditions $W_0 = (0, 1, 0)$ and the initial conditions for $W' $ were $W_0' =(0, 1 + 1\times 10^{-8} , 0)$}
        \label{fig:my_label}
    \end{figure}
\end{question}


\end{document}
